\documentclass{article}

\usepackage{hcar}

\begin{document}

\begin{hcarentry}{Shellac}
\report{Rob Dockins}
\status{beta, maintained}
\makeheader

Shellac is a framework for building read-eval-print style shells.  
Shells are created by declaratively defining a set of shell commands
and an evaluation function.  Shellac supports multiple shell backends,
including a 'basic' backend which uses only Haskell IO primitives and
a full featured 'readline' backend based on the the Haskell readline
bindings found in the standard libraries.

This library attempts to allow users to write shells in a declarative
way and still enjoy the advanced features that may be available from a
powerful line editing package like readline.

Shellac is available from Hackage, as are the related
Shellac-readline, Shellac-editline, and Shellac-compatline
packages.  The readline and editline packages provide
Shellac backends for readline and editline, respectively.
The compatline package is a thin wrapper for either the
readline or editline package, depending on availability
at build-time.

Shellac has been successfully used by several independent projects
and the API is now fairly stable.

\FurtherReading
\url{http://www.cs.princeton.edu/~rdockins/shellac/home}
\end{hcarentry}

\end{document}
