\documentclass{article}

\usepackage{hcar}

\begin{document}

\begin{hcarentry}{Shellac}
\report{Rob Dockins}
\status{beta, maintained}
\makeheader

Shellac is a framework for building read-eval-print style shells.  
Shells are created by declaratively defining a set of shell commands
and an evaluation function.  Shellac supports multiple shell backends,
including a 'basic' backend which uses only Haskell IO primitives and
a full featured 'readline' backend based on the the Haskell readline
bindings found in the standard libraries.

This library attempts to allow users to write shells in a declarative
way and still enjoy the advanced features that may be available from a
powerful line editing package like readline. 

Shellac is avaliable from Hackage, as is the related
Shellac-readline package.

Shellac has been sucessfully used by several independent projects
and the API is now fairly stable.  I will likely be releasing an
offically ``stable'' version in the not-too-distant future.
I anticipate few changes from the current version.

\FurtherReading
\begin{itemize}
\item \url{http://hackage.haskell.org/cgi-bin/hackage-scripts/package/Shellac}
\item \url{http://hackage.haskell.org/cgi-bin/hackage-scripts/package/Shellac-readline}
\end{itemize}
\end{hcarentry}

\end{document}
